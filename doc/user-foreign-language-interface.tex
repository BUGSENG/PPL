% Copyright (C) 2001-2008 Roberto Bagnara <bagnara@cs.unipr.it>
%
% This document describes the Parma Polyhedra Library (PPL).
%
% Permission is granted to copy, distribute and/or modify this document
% under the terms of the GNU Free Documentation License, Version 1.2
% or any later version published by the Free Software Foundation;
% with no Invariant Sections, no Front-Cover Texts, and no Back-Cover Texts.
% The license is included, in various formats, in the `doc' subdirectory
% of each distribution of the PPL in files called `fdl.*'.
%
% The PPL is free software; you can redistribute it and/or modify it
% under the terms of the GNU General Public License as published by the
% Free Software Foundation; either version 3 of the License, or (at your
% option) any later version.  The license is included, in various
% formats, in the `doc' subdirectory of each distribution of the PPL in
% files are called `gpl.*'.
%
% The PPL is distributed in the hope that it will be useful, but WITHOUT
% ANY WARRANTY; without even the implied warranty of MERCHANTABILITY or
% FITNESS FOR A PARTICULAR PURPOSE.  See the GNU General Public License
% for more details.
%
% For the most up-to-date information see the Parma Polyhedra Library
% site: http://www.cs.unipr.it/ppl/ .

\documentclass[a4paper]{article}
\usepackage{a4wide}
\usepackage{makeidx}
\usepackage{fancyhdr}
\usepackage{graphicx}
\usepackage{multicol}
\usepackage{float}
\usepackage{textcomp}
\usepackage{alltt}
\usepackage{times}
\usepackage{ifpdf}
\ifpdf
\usepackage[pdftex,
            pagebackref=true,
            colorlinks=true,
            linkcolor=blue,
            unicode
           ]{hyperref}
\else
\usepackage[ps2pdf,
            pagebackref=true,
            colorlinks=true,
            linkcolor=blue,
            unicode
           ]{hyperref}
\usepackage{pspicture}
\fi
\usepackage[utf8]{inputenc}
\usepackage{doxygen}
<PPL_SED_USEPACKAGE_OCAMLDOC>
\usepackage{ppl}
\makeindex
\setcounter{tocdepth}{2}
\renewcommand{\footrulewidth}{0.4pt}

\rfoot[\fancyplain{}{\scriptsize The Parma Polyhedra Library <PPL_SED_LANGUAGE_NAME> Language Interface User's Manual (version $projectnumber). See \url{http://www.cs.unipr.it/ppl/} for more information.}]{}
\lfoot[]{\fancyplain{}{\scriptsize The Parma Polyhedra Library <PPL_SED_LANGUAGE_NAME> Language Interface User's Manual (version $projectnumber). See \url{http://www.cs.unipr.it/ppl/} for more information.}}

\begin{document}
\title{
The Parma Polyhedra Library \\
<PPL_SED_LANGUAGE_NAME> Language Interface \\
User's Manual\thanks{This work
  has been partly supported by:
  University of Parma's FIL scientific research project (ex 60\%)
    ``Pure and Applied Mathematics'';
  MURST project
    ``Automatic Program Certification by Abstract Interpretation'';
  MURST project
    ``Abstract Interpretation, Type Systems and Control-Flow Analysis'';
  MURST project
    ``Automatic Aggregate- and Number-Reasoning for Computing: from
      Decision Algorithms to Constraint Programming with Multisets,
      Sets, and Maps'';
  MURST project
    ``Constraint Based Verification of Reactive Systems'';
  MURST project
    ``Abstract Interpretation: Design and Applications'';
  EPSRC project
    ``Numerical Domains for Software Analysis'';
  EPSRC project
    ``Geometric Abstractions for Scalable Program Analyzers''.
  } \\
(version $projectnumber)
}
\author{
Roberto Bagnara\thanks{bagnara@cs.unipr.it,
  Department of Mathematics, University of Parma, Italy.} \\
Patricia M. Hill\thanks{hill@comp.leeds.ac.uk,
  School of Computing, University of Leeds, U.K.} \\
Enea Zaffanella\thanks{zaffanella@cs.unipr.it,
  Department of Mathematics, University of Parma, Italy.}
}
\maketitle

\newpage
Copyright \copyright\ 2001--2008 Roberto Bagnara (bagnara@cs.unipr.it).

This document describes the Parma Polyhedra Library (PPL).

Permission is granted to copy, distribute and/or modify this document
under the terms of the GNU Free Documentation License, Version 1.2
or any later version published by the
\href{http://www.fsf.org}{Free Software Foundation};
with no Invariant Sections, no Front-Cover Texts, and no Back-Cover Texts.
A copy of the license is included in the section entitled
``\hyperlink{GFDL_GFDL}{GNU Free Documentation License}''.

The PPL is free software; you can redistribute it and/or modify it
under the terms of the GNU General Public License as published by the
\href{http://www.fsf.org}{Free Software Foundation}; either version 3
of the License, or (at your option) any later version.
A copy of the license is included in the section entitled
``\hyperlink{GPL_GPL}{GNU GENERAL PUBLIC LICENSE}''.

The PPL is distributed in the hope that it will be useful, but WITHOUT
ANY WARRANTY; without even the implied warranty of MERCHANTABILITY or
FITNESS FOR A PARTICULAR PURPOSE.  See the GNU General Public License
for more details.

For the most up-to-date information see the Parma Polyhedra Library
site:
\begin{center}
\href{http://www.cs.unipr.it/ppl/}{\tt http://www.cs.unipr.it/ppl/}
\end{center}

\pagenumbering{roman}
\tableofcontents
\pagenumbering{arabic}
